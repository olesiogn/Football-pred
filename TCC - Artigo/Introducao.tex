\newpage
\section{INTRODUÇÃO}
\pagenumbering{arabic}
O futebol é uma modalidade esportiva disputada por duas equipes de 11 jogadores cada um, cujo objetivo é transpor uma bola entre as extremidades – chamadas de baliza - utilizando basicamente toques com o pé. A baliza, mais conhecida como trave, é um retângulo formado por duas traves verticais, perpendiculares ao solo, e uma paralela ao solo. Entre as traves posiciona-se o goleiro o qual é o único jogador que pode colocar as mãos na bola defendendo o gol. O vencedor da partida será o time que conseguir atingir o objetivo mais vezes no jogo \cite{Sfeir2011}. O fácil entendimento, o baixo custo e a dinâmica empolgante fizeram dessa modalidade uma verdadeira febre mundial. A última copa do mundo jogada na Rússia no ano de 2018 atingiu uma audiência televisiva de mais de 3,5 bilhões de pessoas \cite{FIFAwatch}.

O interesse de tantas pessoas no mundo pelo futebol gera não apenas telespectadores, mas também muitas movimentações financeiras em torno desse esporte. A copa do mundo de 2018 gerou de lucro para a FIFA cerca de 5,35 bilhões de dólares \cite{FIFAlucro}. A movimentação financeira envolve patrocínios a clubes e seleções nacionais, venda de ingressos e produtos licenciados. Além é claro de transações de transferências envolvendo jogadores. Esse fluxo de movimentação financeira atrai uma série de investidores, que sempre visam lucrar. Sistemas computacionais que trabalham com a previsão de resultados e que auxiliam a minimizar os riscos e maximizar os lucros tornam-se então uma importante ferramenta de trabalho para o dia a dia do futebol \cite{Perin2013}.

A utilização de técnicas de inteligência artificial para realizar predição e previsão de resultados de partidas de futebol é um problema bastante explorado na literatura \cite{Perin2013}. Em \citeonline{Martins2017} foi realizado um estudo com a predição de resultados de partidas de futebol envolvendo dados do Campeonato Brasileiro, Espanhol e Inglês. A pesquisa demonstrou a eficiência de algoritmos de Inteligência Artificial para demonstrar quais as características mais relevantes na predição de resultados de partidas de futebol. Tal trabalho deixou em aberto a possibilidade de utilização desses algoritmos para realizar a previsão de resultados de partidas de futebol.

O resultado de uma partida de futebol é o personagem central de inúmeros estudos científicos. Existe um esforço muito grande para melhorar as táticas do jogo e as características de uma determinada equipe. Na literatura, existem estudos que se concentram nas previsões de jogos de futebol \cite{Constantinou2013}. A previsão em partidas de futebol é composta por resultado de uma partida (vitória, empate e derrota) que pode  ser  utilizada para várias finalidades, incluindo apostas. Muitos esforços foram dedicados para a compreensão do futebol a partir da perspectiva dos resultados preditivos. Prever os resultados é um problema difícil devido ao grande número de fatores que devem ser levados em consideração e que nem sempre podem ser representados em valores quantitativos \cite{Hucaljuk2011}. Por exemplo, uma equipe pode dominar completamente uma partida sob alguns aspectos, como o número de finalizações certas, o número de passes certos ou a posse de bola e não conseguir marcar um gol a mais do que a equipe adversária para vencer uma partida \cite{Brooks2016}.

Na literatura é possível encontrar estudos que envolvam predição de resultados de futebol e também de outros esportes coletivos. \citeonline{Ulmer2014} estudaram as técnicas Baseline, Gaussian Naive Bayes, Hidden Markov Model, Multimodal Naive Bayes, SVM, RBF, One vs All SGD para prever resultados utilizando como parâmetros os gols marcados por cada time em 10 temporadas (da temporada 2002-03 à temporada 2011-12) do Campeonato Inglês. \citeonline{Hucaljuk2011} pesquisaram a previsão de resultados para a UEFA Champions League também através de gols marcados com o seguintes algoritmos: Naive Bayes (NB), Redes Bayesianas, Logitboost, K-Nearest Neighbours (KNN), RF e Redes Neurais Artificiais. Com o classificador SVM, \citeonline{Igiri2015} estudou os dados relacionados aos resultados de partidas do Campeonato Inglês.

Com uma quantidade muito maior de características, \citeonline{Owramipur2013} utilizaram dados obtidos através de scout e também de fisiologia para analisar o time de futebol do Barcelona no Campeonato Espanhol. Nesse trabalho, os autores descreveram uma abordagem de redes bayesianas para previsão de resultados de partidas de futebol com o software NETICA. Apesar dos bons resultados obtidos, o modelo considera apenas uma equipe para realizar a previsão do resultado das partidas. \citeonline{Tax2015} usaram os seguintes algoritmos de classificação: CHIRP, LogitBoost, DTNB, FURIA, HiperPipes, J48, Naive Bayes, Perceptron e RF. Para a seleção, eles aplicaram Relief, CfsSubsetEval, e PCA para tentar determinar dentre as 65 características levantadas quais as mais relevantes para aumentar a acurácia dos classificadores. Entretanto, eles não conseguiram determinar quais seriam estas características. O estudo levou em consideração jogos do Campeonato Holandês.

\citeonline{Duarte2015} pesquisaram os classificadores: C5.0, JRip, RF, KNN, SVM e NB para prever as partidas do Campeonato Português com as informações obtidas através de scout e também dados psicológicos dos jogadores das equipes o que, no entanto, não melhorou o desempenho em termos de acurácia. Por esse motivo, é um desafio investigar informações e estratégias que facilitem a previsão dos resultados dos jogos.

Outros estudos foram desenvolvidos para prever o resultado de partidas de futebol utilizando algoritmos de aprendizagem de máquinas. Esses algoritmos são ferramentas que recebem como entrada um conjunto de características e fornece como saída a previsão do resultados (vitória, empate e derrota). Existem algoritmos que podem fornecer uma resposta mais adequada ao problema \cite{Pendharkar2000}.

O objetivo deste trabalho é explorar algoritmos de inteligência artificial como ferramenta de previsão de resultados de partidas de futebol. Como resultado comparar e demonstrar a eficácia para os classificadores utilizados.