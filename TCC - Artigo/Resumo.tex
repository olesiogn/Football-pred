\begin{singlespace}
\textbf{Resumo: }
O futebol é uma modalidade esportiva disputada por duas equipes, cujo objetivo é transpor uma bola entre as
extremidades – chamadas de baliza - utilizando basicamente toques com o pé. O vencedor da partida será o time que conseguir atingir o objetivo mais vezes no jogo. O interesse de tantas pessoas no mundo pelo futebol gera não apenas telespectadores, mas também muitas movimentações financeiras em torno desse esporte. Sistemas computacionais que trabalham com a previsão de resultados e que auxiliam a minimizar os riscos e maximizar os lucros tornam-se então uma importante ferramenta de trabalho para o dia a dia do futebol. O objetivo deste projeto é testar e comparar o desempenho de alguns classificadores na previsão de resultados de partidas de futebol dos campeonatos: Premier League, Serie A e Brasileirão Série A. Os classificadores utilizados foram regressão logística, árvore de decisão, floresta aleatória, knn, svm e mlp. Foram utilizados para divisão dos dados os métodos treino e teste e de validação cruzada. Como resultado tem-se a acurácia de cada classificador.\\
\textbf{Palavras-chave: }
Futebol. Previsão de Resultados. Classificadores. Inteligência Artificial. Aprendizado de Máquina.
\end{singlespace}